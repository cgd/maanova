%%%%%%%%%%%%%%%%%%%%%%%%%%%%
% faq
%%%%%%%%%%%%%%%%%%%%%%%%%%%%
\newpage
\section{Frequently Asked Questions}

\begin{enumerate}
\item Can {\tt R/maanova} run on Windows XP?\\
{\tt R/maanova} runs under R. So if there is a version of R
for your operating system, you can run {\tt R/maanova}.

\item Can {\tt R/maanova} handle missing data?\\
No. {\tt R/maanova} does not tolerant missing, zero or negative
intensity data. Missing data will bring many problems.
A gene with missing data will have a different
experimental design from others. That will cause problem
in permutation tests. So if you have any data missing, you must
manually remove all data associated with that gene
and all its replicates. We suggest you use non-background
subtracted data as input.

\item Can {\tt R/maanova} handle affymetric arrays?\\
Yes. The newest version of {\tt R/maanova} (0.97) works for
N-dye system. But the data visualization and transformation
functions are only working for 2-dye arrays at this time.
So for affymetric data, I suggest you to use some other
packages (such like affy package in BioConduction)
to do the data transformation before loading into {\tt R/maanova}.

\item Can {\tt R/maanova} handle unbalanced design?\\
Yes.

\item Can {\tt R/maanova} output an ANOVA table for a model fitting?\\
Not at this time. We used to output an ANOVA table in the
older version of {\tt R/maanova}. But
as things getting complicated (with multiple factor design,
mixed effects models, etc.), it becomes more and more
difficult to build an ANOVA table. We might think about it
in the future.

\item What does {\tt R/maanova} do with flagged genes?\\
We did not do anything special to the flagged genes.
They will be involved in data transformation and analysis.
We just keep the flag in there so that if some of your
significant genes were flagged, you need to put a
question mark on it because the significance could
be from a scratch on the slide.

\item Why I have problems reading in data?\\
First of all data reading could behave differently
in different operating system. Your data file need
to be simple (ANSI) text file.
Sometimes special characters (such like \#, TAB, etc.)
in the data file will cause problems. If you encounter
that problem, remove the columns contains
a lot of special characters (usually gene description)
and try again. If you still have problem, contact
the software author.

\item Why the estimated effects for a term don't sum to zero?\\
When we build the design matrices we didn't put
sum to zero constraints so that the estimated effects
for a term will not sum to zero. But the relative
values of the estimates (which is the one we care about)
will be correct.

\item Is there a graphical user interface for {\tt R/maanova}?\\
Yes. We are developing a GUI for {\tt R/maanova} using Java.
It will be available in the near future.

\end{enumerate}
